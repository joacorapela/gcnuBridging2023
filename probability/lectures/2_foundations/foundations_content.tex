
\mode<presentation> {

% The Beamer class comes with a number of default slide themes
% which change the colors and layouts of slides. Below this is a list
% of all the themes, uncomment each in turn to see what they look like.

%\usetheme{default}
%\usetheme{AnnArbor}
%\usetheme{Antibes}
%\usetheme{Bergen}
%\usetheme{Berkeley}
%\usetheme{Berlin}
%\usetheme{Boadilla}
%\usetheme{CambridgeUS}
% \usetheme{Copenhagen}
%\usetheme{Darmstadt}
%\usetheme{Dresden}
%\usetheme{Frankfurt}
%\usetheme{Goettingen}
%\usetheme{Hannover}
%\usetheme{Ilmenau}
%\usetheme{JuanLesPins}
%\usetheme{Luebeck}
%\usetheme{Madrid}
%\usetheme{Malmoe}
%\usetheme{Marburg}
%\usetheme{Montpellier}
\usetheme{PaloAlto}
%\usetheme{Pittsburgh}
%\usetheme{Rochester}
%\usetheme{Singapore}
%\usetheme{Szeged}
%\usetheme{Warsaw}

% As well as themes, the Beamer class has a number of color themes
% for any slide theme. Uncomment each of these in turn to see how it
% changes the colors of your current slide theme.

%\usecolortheme{albatross}
%\usecolortheme{beaver}
%\usecolortheme{beetle}
%\usecolortheme{crane}
%\usecolortheme{dolphin}
%\usecolortheme{dove}
%\usecolortheme{fly}
%\usecolortheme{lily}
%\usecolortheme{orchid}
%\usecolortheme{rose}
%\usecolortheme{seagull}
%\usecolortheme{seahorse}
%\usecolortheme{whale}
%\usecolortheme{wolverine}

%\setbeamertemplate{footline} % To remove the footer line in all slides uncomment this line
%\setbeamertemplate{footline}[page number] % To replace the footer line in all slides with a simple slide count uncomment this line

%\setbeamertemplate{navigation symbols}{} % To remove the navigation symbols from the bottom of all slides uncomment this line
}

\hypersetup{colorlinks,citecolor=}
\usepackage{graphicx} % Allows including images
\usepackage{booktabs} % Allows the use of \toprule, \midrule and \bottomrule in tables
\usepackage{natbib}
\usepackage{apalike}
\usepackage{comment}
% \usepackage{enumitem}
% \setlist[itemize]{topsep=0pt,before=\leavevmode\vspace{-1.5em}}
% \setlist[description]{style=nextline}
\usepackage{amsthm}
\usepackage{media9}
% \usepackage{multimedia}

\newtheorem{claim}{Claim}


%----------------------------------------------------------------------------------------
%	TITLE PAGE
%----------------------------------------------------------------------------------------

\title{Foundations of probability theory}

\author{Joaqu\'{i}n Rapela} % Your name
\institute[GCNU, UCL] % Your institution as it will appear on the bottom of every slide, may be shorthand to save space
{
Gatsby Computational Neuroscience Unit\\University College London % Your institution for the title page
}
\date{May 18, 2022} % Date, can be changed to a custom date

\AtBeginSection[]
  {
     \begin{frame}<beamer>
     \frametitle{Contents}
         \tableofcontents[currentsection,hideallsubsections]
     \end{frame}
  }

\begin{document}

\begin{frame}
\titlepage % Print the title page as the first slide
\end{frame}

\begin{frame}
\frametitle{Contents} % Table of contents slide, comment this block out to remove it
\tableofcontents % Throughout your presentation, if you choose to use \section{} and \subsection{} commands, these will automatically be printed on this slide as an overview of your presentation
\end{frame}

\begin{frame}
\frametitle{Main reference} % Table of contents slide, comment this block out to remove it

    I will mainly follow chapters seven \textit{Foundations of probability
    theory} and eight \textit{Conditional probability and Bayes} from
    \citet{tijms12}.

\end{frame}

\section{Foundations of probability theory}

\subsection{Historical notes}

\begin{frame}
\frametitle{Historical notes}

- explain the frequency-based interpretation of probability.

- constructing the mathematical foundations of probability theory has proven to be a long-lasting process of trial an error.  

- the approach of defining probability as relative frequencies of repeatable experiments lead to unsatisfactory theory (why?)
https://www.jstor.org/stable/pdf/20115155.pdf

- the frequency view of probability has a long history that goes back to Aristotle.

- in 1933 the Russian mathematician Andrej Kolmogrov (1903-1987) laid a satisfactory mathematical foundation of probability theory.

He created a set of axioms. Axioms state a number of minimal requirements that the probability objects should satisfy. From these few algorithms all claims of probability can be derived, as we will see.

\end{frame}

\subsection{Axioms of probability theory}

\begin{frame}
\frametitle{Probabilistic foundation}

- sample space

    - examples of finite, countable and uncountable

    - mention the proof by Cantor that the real numbers are not countable

\end{frame}

\begin{frame}
\frametitle{Axioms of probability theory}

- events in countable and uncountable sample spaces (hint about sigma algebras)

- probability measure

    - three axioms

        - explain what an infinite union means

- probability space: sample space + events + probability measure = probability space

- building a probability measure for a finite or countable sample space.

- probability model = sample space + probability measure

\end{frame}

\begin{frame}
    \frametitle{Equally likely outcomes}

- examples

- uncountable sample space

\end{frame}

\subsection{Some basic rules}

\begin{frame}
    \frametitle{Some basic rules}

    \begin{description}

        \item{Rule 1} For any finite number of mutually exclusive events
            $A_1,\ldots,A_n$,

            \begin{align*}
                P(A_1,\ldots,A_n) = P(A_1) + \ldots + P(A_n)
            \end{align*}

        \item{Rule 2} For any event A,

            \begin{align*}
                P(A) = 1 - P(A^c)
            \end{align*}

            where the event $A^c$ consists of all outcomes that are not in $A$.

        \item{Rule 3} For any two events A and B,

            \begin{align*}
                P(A\cup B) = P(A) + P(B) - P(AB)
            \end{align*}

    \end{description}

\end{frame}

\begin{frame}
    \frametitle{Proof of rule 1}

\end{frame}

\begin{frame}
    \frametitle{Proof of rule 2}

\end{frame}

\begin{frame}
    \frametitle{Proof of rule 3}

\end{frame}

\begin{frame}
    \frametitle{Example: Chevalier de Mere to Blaise Pascal}

    - example 7.7 (rule 7-2): Chevalier de Mere to Blaise Pascal 1654

\end{frame}

\begin{frame}
    \frametitle{Example:}

    - example 7.8 (rule 7-3, addition rule, easy)

\end{frame}

\begin{frame}
    \frametitle{Example:}

    - example 7.9 (rule 7-3): uses counting tools (binomial coefficient)

        - wrong, but simple, approach

        - correct, but more complicated, approach

        - sampling approach

\end{frame}

\begin{frame}
    \frametitle{Example:}

    - example 7.10 (rule 7-1, birthday problem, used in example 8.6): uses counting tools (binomial coefficient)

\end{frame}

\section{Conditional probability and Bayes}

\subsection{Conditional probability}

\begin{frame}
    \frametitle{Conditional probability}

- p. 256: good motivation of conditional probability in the cards example

- Definition 8.1

- interpretation of condition probability with relative frequencies

\end{frame}

\begin{frame}
    \frametitle{Example:}

- Example 8.1 (first ask students their intuition, as the problem is counter intuitive)

- do NOT present example 8.2 at this point, as it requires the concept of independence

\end{frame}

\subsection{Assigning probabilities by conditioning}

\begin{frame}
    \frametitle{Assigning probabilities by conditioning}

    \begin{description}

        \item[Rule 4] For any sequence of events $A_1,\ldots,A_n$,

            \begin{align*}
                P(A_1, ..., A_n) = P(A_n|A_{n-1}, ..., A_1) \ldots P(A_1)
            \end{align*}

    \end{description}

\end{frame}

\begin{frame}
    \frametitle{Example:}

    - redo Example 7.9 (solution following Rule 4)

\end{frame}

\begin{frame}
    \frametitle{Example:}

    - probability that it takes 10 or more cards before the first ace appears

\end{frame}

\subsection{Independent events}

\begin{frame}
    \frametitle{Independent events}

- motivation of independence definition with conditional probabilities

- Definition 8.2

\end{frame}

\begin{frame}
    \frametitle{Example}

- Example 8.5

\end{frame}

\begin{frame}
    \frametitle{Example}

- Example 8.6 (uses birthday problem, example 7.10)

\end{frame}

\subsection{Law of conditional probability}

\begin{frame}
    \frametitle{Law of conditional probability}

- example of dice followed by coin tosses

    \begin{description}

        \item{Rule 5} law of conditional probability. Let $A$ be an event that
            can only occur if one of the mutually exclusive events
            $B_1,\ldots,B_n$ occurs. Then

            \begin{align*}
                P(A) = P(A|B_1) P(B_1) + \ldots + P(A|B_n) P(B_n)  
            \end{align*}

    \end{description}

\end{frame}

\begin{frame}
    \frametitle{Example}

- example 8.6: tour the France (difficult!)

\end{frame}

\subsection{Baye's rule in odds form}

\begin{frame}
    \frametitle{Bayes rule in odds form}

- true/false hypothesis

    \begin{description}

        \item{Rule 6} The posterior probability $P(H|E)$ satisfies

            \begin{align*}
                \frac{P(H|E)}{P(\bar{H}|E)} = \frac{P(H)}{P(\bar{H})}\frac{P(E|H)}{P(E|\bar{H})}
            \end{align*}

    \end{description}

- interpretation of rule 6

    - avoid need of P(E)

    - prior odds + likelihood ratio or Bayes factor

    - prior odds update with new evidence

    - sequential update (mention Bayesian linear regression)

\end{frame}

\begin{frame}
    \frametitle{Example}

- example 8.8

\end{frame}

\begin{frame}
    \frametitle{Example}

- example 8.11

    - add to the problem statement:

        - in 1992, 4936 women were murdered in the US, of which roughly 1430 were murdered by their (ex)husbands or boyfriends

        - 5\% of the married women in the US have at some point been physically abused by their husbands.

        - assume that a woman who has been murdered by some other than her husband had the same same chance of being abused by her husband as a randomly selected woman

        - Alan Dershowitz admitted that a substantial percentage of the husbands who murder their wives, previous to the murder, also physically abuse their wives. Given this statement, we assume that the proability that a husband physically abused his wife, given that he killed her, is 50 percent.

\end{frame}

\subsection{Bayesian inference -- discrete case}

\begin{frame}
    \frametitle{Bayesian inference -- discrete case}

- explain posterior sequential update

\end{frame}

\begin{frame}
    \frametitle{Example}

- example 8.13 (solve it analytically and by sampling)

\end{frame}

\begin{frame}
    \frametitle{References}

    \tiny{
        \bibliographystyle{apalike}
        \bibliography{probability}
    }
\end{frame}

\begin{comment}
\end{comment}

\end{document}

