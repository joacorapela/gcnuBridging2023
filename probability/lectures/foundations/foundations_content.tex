
\mode<presentation> {

% The Beamer class comes with a number of default slide themes
% which change the colors and layouts of slides. Below this is a list
% of all the themes, uncomment each in turn to see what they look like.

%\usetheme{default}
%\usetheme{AnnArbor}
%\usetheme{Antibes}
%\usetheme{Bergen}
%\usetheme{Berkeley}
%\usetheme{Berlin}
%\usetheme{Boadilla}
%\usetheme{CambridgeUS}
% \usetheme{Copenhagen}
%\usetheme{Darmstadt}
%\usetheme{Dresden}
%\usetheme{Frankfurt}
%\usetheme{Goettingen}
%\usetheme{Hannover}
%\usetheme{Ilmenau}
%\usetheme{JuanLesPins}
%\usetheme{Luebeck}
%\usetheme{Madrid}
%\usetheme{Malmoe}
%\usetheme{Marburg}
%\usetheme{Montpellier}
\usetheme{PaloAlto}
%\usetheme{Pittsburgh}
%\usetheme{Rochester}
%\usetheme{Singapore}
%\usetheme{Szeged}
%\usetheme{Warsaw}

% As well as themes, the Beamer class has a number of color themes
% for any slide theme. Uncomment each of these in turn to see how it
% changes the colors of your current slide theme.

%\usecolortheme{albatross}
%\usecolortheme{beaver}
%\usecolortheme{beetle}
%\usecolortheme{crane}
%\usecolortheme{dolphin}
%\usecolortheme{dove}
%\usecolortheme{fly}
%\usecolortheme{lily}
%\usecolortheme{orchid}
%\usecolortheme{rose}
%\usecolortheme{seagull}
%\usecolortheme{seahorse}
%\usecolortheme{whale}
%\usecolortheme{wolverine}

%\setbeamertemplate{footline} % To remove the footer line in all slides uncomment this line
%\setbeamertemplate{footline}[page number] % To replace the footer line in all slides with a simple slide count uncomment this line

%\setbeamertemplate{navigation symbols}{} % To remove the navigation symbols from the bottom of all slides uncomment this line
}

\hypersetup{colorlinks,citecolor=}
\usepackage{graphicx} % Allows including images
\usepackage{booktabs} % Allows the use of \toprule, \midrule and \bottomrule in tables
\usepackage{natbib}
\usepackage{apalike}
\usepackage{comment}
% \usepackage{enumitem}
% \setlist[itemize]{topsep=0pt,before=\leavevmode\vspace{-1.5em}}
% \setlist[description]{style=nextline}
\usepackage{amsthm}
\usepackage{media9}
% \usepackage{multimedia}

\newtheorem{claim}{Claim}


%----------------------------------------------------------------------------------------
%	TITLE PAGE
%----------------------------------------------------------------------------------------

\title{Foundations of probability theory}

\author{Joaqu\'{i}n Rapela} % Your name
\institute[GCNU, UCL] % Your institution as it will appear on the bottom of every slide, may be shorthand to save space
{
Gatsby Computational Neuroscience Unit\\University College London % Your institution for the title page
}
\date{May 18, 2022} % Date, can be changed to a custom date

\AtBeginSection[]
  {
     \begin{frame}<beamer>
     \frametitle{Contents}
         \tableofcontents[currentsection,hideallsubsections]
     \end{frame}
  }

\begin{document}

\begin{frame}
\titlepage % Print the title page as the first slide
\end{frame}

\begin{frame}
\frametitle{Contents} % Table of contents slide, comment this block out to remove it
\tableofcontents % Throughout your presentation, if you choose to use \section{} and \subsection{} commands, these will automatically be printed on this slide as an overview of your presentation
\end{frame}

\section{Historical notes}

\section{Probability model}

\section{Conditional probability}

\section{Random variables}

\section{Expected value and law of large numbers}

\begin{frame}
    \frametitle{The best-choice problem}

    \footnotesize
    \hfill\citet{tijms12}

    \textit{
        %
        Your friend proposes the following wager: twenty people are requested
        to write a number in a piece of paper. They write any number they like,
        no matter how high. You fold up the twenty pieces of paper and place
        them randomly on a tabletop.
        %
        Your friends task is to single out the paper with the largest number.
        He opens the papers one by one. Each time he opens one, he must decide
        whether to stop with that one or go on to open another one. Once a
        paper is opened, your friend cannot go back to any of the previously
        opened paper.
        %
        He pays you one dollar if he does not identify the paper with the
        highest number on it, otherwise you pay him five dollars. Do you take
        the wager?  }

    \textit{
        %
        What would you say to a similar wager with 100 people where your friend
        pays you one dollar if he does not find the paper with the highest
        number and you pay him ten dollars otherwise?
        %
    }
        
    \textit{ 
        %
        Your friend will use the following strategy. He will allow the first half
        of papers to pass through his hands, and remember the highest number that
        has appeared. As he process the second half, he chooses the first paper
        showing a number higher than the one he remembered in the first half.
        %
    }

\end{frame}

\begin{frame}
    \frametitle{References}

    \tiny{
        \bibliographystyle{apalike}
        \bibliography{probability}
    }
\end{frame}

\end{document}

